\documentclass{article}

\usepackage{Sweave}
\begin{document}
\input{10-CaricamentoFile-concordance}
\section{Caricamento dati da file}
R da la possibilità di caricare dati che sono salvati in differenti formati:
\begin{itemize}
  \item txt
  \item csv
  \item xlsx
  \item RData
\end{itemize}

\subsection{Caricamento da txt}
Per caricare i dati da un file con estensione \textbf{txt} è possibile utilizzare
il comando \texttt{read.delim2} al quale è necessario specificare alcune opzioni:
\begin{itemize}
  \item il nome del file;
  \item \texttt{header} da specificare a \texttt{TRUE} oppure \texttt{FALSE} a 
  seconda che si voglia che la prima riga venga interpretata come intestazione;
  \item \texttt{sep} per specificare il separatore tra i dati (tra virgolette);
  \item \texttt{dec} per specificare il carattere che divide i decimali.
\end{itemize}

\begin{Schunk}
\begin{Sinput}
> read.delim2("file.txt", header = TRUE, sep = "\t", dec = ",")