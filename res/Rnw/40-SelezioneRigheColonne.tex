\documentclass{article}

\usepackage{Sweave}
\begin{document}
\input{40-SelezioneRigheColonne-concordance}
\section{Selezione di righe e colonne}
Quando sono stati caricati correttamente i dati spesso è necessaria solamente
una parte di essi, per la visualizzazione oppure poichè alcuni sono più 
rilevanti di altri. 

Per fare ciò R permette di utilizzare la notazione con le parentesi quadre.
In pratica presa una variabile con cui possiamo accedere ai nostri dati, 
per esempio \texttt{v}, con la notazione \textit{v[x, y]} possiamo accedere
al dato che si trova alla riga \textit{x} della colonna \textit{y}. Molto 
spesso succede che sia necessario prendere tutte le colonne di una certa riga
o, viceversa, tutte le righe di una certa colonna. In questo caso è sufficiente
utilizzare rispettivamente la notazione \texttt{v[x, ]} e \texttt{v[, y]}.

R inoltre permette di selezionare più righe (o colonne allo stesso modo). Per 
fare questo basta indicare tra le parentesi quadre \texttt{x1:x2}, dove 
\texttt{x1} rappresenta la riga di inizio e \texttt{x2}

\begin{Schunk}
\begin{Sinput}
> library(MASS)
> # Carico una libreria che contiene alcuni dataset
> data(Boston)
> # Caricamento dal dataset Boston
> v<-Boston
> # Associo a v il dataset
> v[3,6]
> v[3,]
> v[,6]
> v[3:9,6]
> v[,6:12]
\end{Sinput}
\end{Schunk}

\end{document}
