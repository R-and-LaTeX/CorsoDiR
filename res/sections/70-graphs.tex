\chapter{Grafici}

Arriviamo a esplorare una delle molteplici punte di diamante che questo
linguaggio possiede. R, infatti, non è solamente un ottimo linguaggio per i
calcoli di natura statistica, ma permette anche di creare diverse tipologie di
grafici, partendo da quelli semplici per arrivare a quelli più avanzati, con
didascalia e legenda per esempio.
I grafici che vedremo sono:
\begin{itemize}
 \item Istogramma
 \item Torta
 \item Dispersione
 \item Boxplot
\end{itemize}

Alla fine esploreremo comandi di contorno per migliorare i nostri grafici.

\section{Istogramma}

Per creare degli istogrammi il comando da usare è \texttt{hist()}. Vediamo
subito un esempio:

\input{res/Rnw/70-graphs1.Rnw.include.generated}

Oltre a \texttt{xlab} è anche presente \texttt{ylab} per inserire
\textit{label} nell'asse delle \texttt{y}.

\section{Torta}

I grafici a torta sono utili per poter rappresentare dati in maniera visuale e
semplice da capire per il lettore.
Vediamo come funziona:

\input{res/Rnw/70-graphs2.Rnw.include.generated}

\section{Dispersione}

\section{Boxplot}

\section{Altri comandi secondari}

% TODO spiegare i comandi abline, mfrow e par
