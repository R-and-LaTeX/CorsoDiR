\chapter{Grafici}

Arriviamo a esplorare una delle molteplici punte di diamante che questo
linguaggio possiede. R, infatti, non è solamente un ottimo linguaggio per i
calcoli di natura statistica, ma permette anche di creare diverse tipologie di
grafici, partendo da quelli semplici per arrivare a quelli più avanzati, con
didascalia e legenda per esempio.
I grafici che vedremo sono:
\begin{itemize}
 \item Istogramma
 \item Torta
 \item Dispersione
 \item Boxplot
\end{itemize}

Alla fine esploreremo comandi di contorno per migliorare i nostri grafici.

\section{Istogramma}

Per creare degli istogrammi il comando da usare è \texttt{hist()}. Vediamo
subito un esempio:

\input{res/Rnw/70-graphs1.Rnw.include.generated}

Oltre a \texttt{xlab} è anche presente \texttt{ylab} per inserire
\textit{label} nell'asse delle \texttt{y}.

\section{Torta}

I grafici a torta sono utili per poter rappresentare dati in maniera visuale e
semplice da capire per il lettore.
Vediamo come funziona:

\input{res/Rnw/70-graphs2.Rnw.include.generated}

Come possiamo notare, abbiamo utilizzato delle nuove funzionalità di R.

\paragraph*{\texttt{sapply()}} Questa funzione, applicata ad un dataset, serve
per applicare la funzione passata come secondo argomento per tutte le variabili
che il dataset che contiene.

\paragraph*{\texttt{levels()}} La funzionalità di \texttt{levels()}, quando
applicata ad una variabile qualitativa, è quella di visualizzare i vari livelli
della variabile stessa. È comodo per conoscere in anticipo i nomi della
categorizzazione, così da poterli riscrivere al momento della della creazione
del grafico a torta.

\paragraph*{\texttt{c()}} \texttt{c()} è un'abbreviazione per
\textit{concatenate}, ed è utile in quanto ogni volta che si passano argomenti
a questa funzione questi vengono concatenati e viene ritornato un
\textit{array} con essi contenuti. In \texttt{R} è utile in quanto alcune
funzioni accettano \textit{array} come parametri, e \texttt{c()} permette di
crearli facilmente senza salvare altre variabili.

\section{Dispersione}

I grafici a dispersione permettono di rappresentare un insieme di punti in un
grafico. In \texttt{R} questi si chiamano \textit{Scatterplot}, e il comando
per poterli disegnare è tramite la funzione \texttt{plot()}, a cui andranno
passate due variabili. Vediamo un esempio:

\input{res/Rnw/70-graphs3.Rnw.include.generated}

Se volessimo vedere un grafico di dispersione dei punti tra tutte le variabili
nel nostro \textit{dataset} possiamo utilizzare la funzione \texttt{plot()}
direttamente sul dataset. Usando questo comando però potrebbe richiedere una
buona quantità di risorse al nostro computer, quindi quando diamo questo
comando prestiamo attenzione\footnote{I grafici poi potrebbero risultare molto
piccoli e poco visibili.}.

\section{Boxplot}

A volte potrebbe essere utile vedere in maniera grafica se due variabili nel 
nostro \textit{dataset} sono correlate tra di loro. Oltre a questo scopo, un 
boxplot di una sola variabile ci permette di capire la distribuzione dei punti, 
evidenziando dei possibili \textit{outlier}.

Il comando per creare un boxplot è \texttt{boxplot()}.

\subsection{Boxplot con una variabile}

Fare un boxplot con una variabile è molto semplice:

\input{res/Rnw/70-graphs5.Rnw.include.generated}

\subsection{Boxplot con più variabili}

Come detto, un \texttt{Boxplot} è utile per poter confrontare graficamente la 
correlazione tra due variabili. Ad esempio:

\input{res/Rnw/70-graphs6.Rnw.include.generated}

\section{Altri comandi secondari}

Oltre ai comandi visti fin'ora per eseguire grafici, ne esistono di aggiuntivi
che vedremo qui di seguito.

\subsection{\texttt{abline()}}

Questa funzione ha la comoda funzionalità di disegnare linee in un grafico già
esistente. Vediamo:

\input{res/Rnw/70-graphs4.Rnw.include.generated}

Come possiamo notare, per esempio in questo caso \texttt{h} indica di disegnare
una retta orizzontale con l'altezza che abbiamo desiderato.
È anche possibile selezionare un determinato colore della retta, tramite
l'opzione \texttt{col}.

\subsection{\texttt{par()} e \texttt{mfrow()}}

Potrebbe essere utile dover confrontare facilmente due grafici. Purtroppo 
quando si crea un nuovo grafico \texttt{R} cancella quello precedente, e quindi 
è necessario dire al linguaggio in qualche maniera che ne vogliamo tenere due o 
più nella stessa finestra.
Per fare ciò esiste il comando \texttt{par()}, che insieme a \texttt{mfrow()} 
permette di creare ``una griglia'' di immagini affiancate.
Ad esempio, il comando

\input{res/Rnw/70-graphs7.Rnw.include.generated}

Permette di dividere lo schermo in una griglia $n \cdot n$.

\section{Salvare su file}

Potrebbe essere comodo salvare su file i grafici ottenuti, per poterli 
conservare o riutilizzare in altri progetti.
Il comando è \texttt{png()}, che permette di salvare un png dell'immagine 
attualmente disegnata a schermo su file.
Vediamo come funziona:

\input{res/Rnw/70-graphs8.Rnw.include.generated}

