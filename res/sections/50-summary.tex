\chapter{Comandi base}

\section{Summary}

Quando abbiamo un sacco di dati ci potrebbe essere utile avere una visione
d'insieme. Qual è la media degli studenti del corso? Più o meno di ventordici?
Chi è il più scansafatiche? E il più diligente? Sono presenti elementi che si
discostano fortemente dalla media\footnote{Detti anche \textit{outlier}}?
\texttt{R} fornisce uno strumento per ognugna di queste domande, ma siccome in
media gli informatici son pigri\footnote{Chi ha voglia in fondo di premere tutti
quei tasti sulla tastiera? Una faticaccia!} esiste un comando che li riassume
tutti, che si chiama \texttt{summary()}\footnote{Del resto chi se lo poteva
immaginare? Manco fosse il titolo della sezione...}, che fornisce le seguenti
indicazioni:
\begin{itemize}
 \item Il valore \textbf{minimo} di ogni variabile
 \item Il \textbf{primo quartile} di ogni variabile
 \item La \textbf{mediana} di ogni variabile nel dataset
 \item Le \textbf{media} di ogni variabile nel dataset
 \item Il \textbf{terzo quartile} di ogni variabile
 \item Il valore \textbf{massimo} di ogni variabile
\end{itemize}

Vediamo subito un esempio:

\input{res/Rnw/50-summary1.Rnw.include.generated}

Come possiamo notare, questo è utile, ma se abbiamo dataset grandi con un sacco
di variabili magari può essere una rottura, specialmente per la quantità di
output che viene mostrata a schermo. Per ridurre possiamo visualizzare
solamente una parte di esso.

\input{res/Rnw/50-summary2.Rnw.include.generated}

\section{Smontando Summary}

È giunto il momento di ``smontare'' \texttt{summary()} e vedere che meraviglie
nasconde sotto il cofano. Prima abbiamo detto che è solo un comando che ne
racchiude altri. Questi sono:
\begin{itemize}
 \item Per il minimo, il comando è \texttt{min()}
 \item Per i quartili il comando è \texttt{quantile()} che ci farà
vedere tutti i quartili (quelli a $\frac{0}{4}, \frac{1}{4}, \frac{2}{4},
\frac{3}{4}, \frac{4}{4}$)
 \item Per la mediana il comando è \texttt{median()}
 \item Per la media il comando è \texttt{mean()}
 \item Il massimo si ricava con \texttt{max()}
\end{itemize}

Per parafrasare i libri di matematica, la dimostrazione che le seguenti
funzioni diano gli stessi esatti risultati di valori che vengono stampati a
video da \texttt{summary()} è lasciata al lettore.
