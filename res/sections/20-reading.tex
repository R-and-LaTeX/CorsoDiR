% Eliminare colonne vuote
% Verificare che variabili qualitative lo siano davvero
\chapter{Leggere correttamente i dati}

A volte può capitare che i dati che abbiamo non siano completi, o che ci sia la
necessità di verificare che alcune variabili abbiano le proprietà desiderate.
Per fare ciò R mette a disposizione dei semplici comandi che verranno spiegati
di seguito.

\section{Filtrare dati letti}

Questa funzionalità è particolarmente utile quanto ci ritroviamo con dei dati
che possono contenere delle celle ad esempio vuote o con dati che sappiamo non
essere corretti.

Il comando per vedere se ci sono dati vuoti in un \textit{dataset} è
\texttt{is.na()}, dove, come parametro della funzione, dobbiamo passare i
nostri dati. In aggiunta a questo comando possiamo aggiungere \texttt{sum()},
che permette di contare vari elementi. Combinando \texttt{sum()} con
\texttt{is.na()} avremo il numero di celle che contengono dati vuoti.
Proviamo con un esempio:

\input{res/Rnw/20-reading.Rnw.include.generated}

Per esempio con il \textit{dataset} R ci dice che ci sono $59$ dati nulli.

Con la statistica ci ha insegnato (o avrebbe dovuto) che le variabili possono
considerarsi di due tipi: qualitative o quantitative. R è in grado di eseguire
questa distinzione tra le variabili, ma a volte potrebbe sbagliare.
Fortunatamente il linguaggio stesso ci da la possibilità di controllare il tipo
delle variabili. Per fare ciò dobbiamo usare il comando:

