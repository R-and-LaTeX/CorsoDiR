% Eliminare colonne vuote
% Verificare che variabili qualitative lo siano davvero
\chapter{Leggere correttamente i dati}

A volte può capitare che i dati che abbiamo non siano completi, o che ci sia la
necessità di verificare che alcune variabili abbiano le proprietà desiderate.
Per fare ciò R mette a disposizione dei semplici comandi che verranno spiegati
di seguito.

\section{Filtrare dati letti}

Questa funzionalità è particolarmente utile quanto ci ritroviamo con dei dati
che possono contenere delle celle ad esempio vuote o con dati che sappiamo non
essere corretti.

Il comando per vedere se ci sono dati vuoti in un \textit{dataset} è
\texttt{is.na()}, dove, come parametro della funzione, dobbiamo passare i
nostri dati. In aggiunta a questo comando possiamo aggiungere \texttt{sum()},
che permette di contare vari elementi. Combinando \texttt{sum()} con
\texttt{is.na()} avremo il numero di celle che contengono dati vuoti.
Proviamo con un esempio:

\input{res/Rnw/20-reading1.Rnw.include.generated}

Per esempio con il \textit{dataset} R ci dice che ci sono $59$ dati nulli.
Come possiamo eliminare queste celle vuote? Un metodo per farlo è usando la
seguente funzione:

\input{res/Rnw/20-reading7.Rnw.include.generated}

Questa funzione restituisce il dataset che gli passiamo come argomento, ma
senza i dati vuoti. Se vogliamo salvare il tutto dobbiamo eseguire
l'assegnazione ad una nuova variabile, e per fare ciò dobbiamo usare il simbolo
di assegnazione, che in R è \texttt{<-}.

\section{Variabili quantitative e qualitative}

Come la statistica ci ha insegnato (o avrebbe dovuto) le variabili possono
considerarsi di due tipi: qualitative o quantitative. R è in grado di eseguire
questa distinzione tra le variabili, ma a volte potrebbe sbagliare, soprattutto
quando si caricano dei dati da file\footnote{Questo si verifica spesso se
stiamo andando a creare un modello statistico.}.
Fortunatamente il linguaggio stesso ci da la possibilità di controllare il tipo
delle variabili. Per fare ciò dobbiamo usare il comando:

\input{res/Rnw/20-reading2.Rnw.include.generated}

Questa funzione ci dice se una variabile è qualitativa (rispondendoci con
\texttt{TRUE}) o quantitativa (rispondendo con \texttt{FALSE}). Se vogliamo far
interpretare una variabile quantitativa come qualitativa usiamo la funzione:

\input{res/Rnw/20-reading3.Rnw.include.generated}

Questo causerà la reinterpretazione della variabile da parte di R da quello che
era precedentemente a qualitativa.

Se vogliamo fare gli alternativi invece possiamo usare la seguente funzione per
verificare che una variabile sia quantitativa:

\input{res/Rnw/20-reading4.Rnw.include.generated}

Per convertire una variabile in una variabile quantitativa usiamo il comando:

\input{res/Rnw/20-reading5.Rnw.include.generated}

\section{Esempio finale}

Facendo vedere un esempio completo avremo quindi:

\input{res/Rnw/20-reading6.Rnw.include.generated}
